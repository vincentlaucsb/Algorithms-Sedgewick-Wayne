\documentclass[]{article}
\usepackage{color}

%opening
\title{2.1 Elementary Sorts Exercises}
\author{Vincent La}

\begin{document}

\maketitle

\begin{enumerate}
	\item[2.1.1] Show, in the style of the example trace with Algorithm 2.1, how selection sort sorts the array E A S Y Q U E S T I O N.
	
	\begin{tabular}{c c c c c c c c c c c c c c}
		i & min & 0 & 1 & 2 & 3 & 4 & 5 & 6 & 7 & 8 & 9 & 10 & 11\\
		\hline
		   &   & E & A & S & Y & Q & U & E & S & T & I & O & N \\
		0  & 1 & E & \color{red}A & S & Y & Q & U & E & S & T & I & O & N \\	
		1  & 1 & A & \color{red}E & S & Y & Q & U & E & S & T & I & O & N \\
		2  & 6 & A & E & S & Y & Q & U & \color{red}E & S & T & I & O & N \\
		3  & 9 & A & E & E & Y & Q & U & S & S & T & \color{red}I & O & N \\
		4  & 11 & A & E & E & I & Q & U & S & S & T & Y & O & \color{red}N \\
		5  & 10 & A & E & E & I & N & U & S & S & T & Y & \color{red}O & Q \\
		6  & 11 & A & E & E & I & N & O & S & S & T & Y & U & \color{red}Q \\
		7  & 7  & A & E & E & I & N & O & Q & \color{red}S & T & Y & U & S \\
		8  & 11 & A & E & E & I & N & O & Q & S & T & Y & U & \color{red}S \\
		9  & 11 & A & E & E & I & N & O & Q & S & S & Y & U & \color{red}T \\
		10 & 10 & A & E & E & I & N & O & Q & S & S & T & \color{red}U & Y \\
		   &    & A & E & E & I & N & O & Q & S & S & T & U & Y \\
	\end{tabular}
	
	\item[2.1.2] The maximum number of exchanges involving a specific item is $N$ exchanges. For example, take a list that is already sorted and then add to the beginning an item that is greater in value than the rest of the list. Specifically, consider the list Y A B C D.
	
	\begin{tabular}{c c c c c c c}
		i & min & 0 & 1 & 2 & 3 & 4 \\
		\hline
		  &     & Y & A & B & C & D \\
		0 & 1   & Y & \color{red}A & B & C & D \\
		1 & 2   & A & Y & \color{red}B & C & D \\
		2 & 3   & A & B & Y & \color{red}C & D \\
		3 & 4   & A & B & C & Y & \color{red}D \\
		4 & 4   & A & B & C & D & \color{red}Y \\
		  &     & A & B & C & D & Y \\
	\end{tabular}
	
	Here, we can see Y was exchanged $N$ times. Furthermore, because a selection sort performs one exchange for every array index, then there are at most $N$ exchanges for an array of length $N$.
	
	\bigskip
	
	On the other hand, because there are $N$ items and $N$ exchanges then on average each item gets exchanged once. (Is this right)
	
	\item[2.1.3] Give an example of an array of $N$ items that maximizes the number of times the test a[j] $<$ a[min] succeeds (and, therefore, min gets updated) during the operation of selection sort.
	
	\paragraph{Answer} Hypothetically, the most that a[j] $<$ a[min] can succeed is every time it is called. In fact, a list in descending order does just that. For example, consider the list E D C B A. Before the first exchange, the following compares are made:
		\begin{enumerate}
			\item D $<$ E, so $min = 1$
			\item C $<$ D, so $min = 2$
			\item B $<$ C, so $min = 3$
			\item A $<$ B, so $min = 4$
		\end{enumerate}
		
	(Expand on this answer later)
\end{enumerate}

\end{document}

