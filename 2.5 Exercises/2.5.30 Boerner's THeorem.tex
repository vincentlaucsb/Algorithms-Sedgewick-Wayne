\documentclass[11pt]{article}
\usepackage[margin=1.25in]{geometry}

\usepackage{graphicx,tikz}
\usepackage{amsmath,amsthm,mathdots}
\usepackage{amsfonts}
\usepackage{amssymb}
\usepackage{boondox-cal}
\title{ }

\newtheorem*{thm}{Theorem}
\newcommand{\spoly}[1]{\mathcal{P}\left(#1\right)}
\newcommand{\norm}[1]{\left\lVert#1\right\rVert}
\newcommand{\iprod}[2]{\left\langle#1,#2\right\rangle}
\newcommand{\tallb}[1]{\left[#1\right]}
\newcommand{\tallp}[1]{\left(#1\right)}
\newcommand{\abs}[1]{\left|#1\right|}
\newcommand{\overbar}[1]{\mkern 1.5mu\overline{\mkern-1.5mu#1\mkern-1.5mu}\mkern 1.5mu}
\newcommand{\spn}[1]{\text{span}\tallp{#1}}
\newcommand{\nll}[1]{\text{null}\tallp{#1}}
\newcommand{\grams}[1]{\frac{#1}{\norm{#1}}}
\newcommand{\set}[2]{\{\left #1 \text{ } | \text{ } #2 \right\}}
\newcommand{\range}[1]{\text{range }#1}

\title{Boerner's Theorem}
\author{Vincent La}

\begin{document}
\maketitle
\date

\paragraph{True or False} If you sort each column of a matrix, then sort each row, the columns are still sorted. Justify your answer.

\bigskip

True.

\begin{proof}
    Given any matrix, suppose we sort each column and then sort each row. For this case, suppose we sort in ascending order. Now, suppose for a contradiction that the columns are no longer sorted. This implies we have a matrix in the form
    
    \[\begin{bmatrix}
    * &  &   & \\
      &  & b & \geq b \\
      &  & a & \geq a \\
    \end{bmatrix}\]
    
    where $b$ is some number greater than $a$. Because the rows are sorted, we know that $a$ is greater than or equal to all numbers to the left of it. Furthermore, we know that $b$ was moved during the sort by row operation. 
    
    
\end{proof}



\end{document}